% Options for packages loaded elsewhere
\PassOptionsToPackage{unicode}{hyperref}
\PassOptionsToPackage{hyphens}{url}
%
\documentclass[
]{article}
\usepackage{amsmath,amssymb}
\usepackage{iftex}
\ifPDFTeX
  \usepackage[T1]{fontenc}
  \usepackage[utf8]{inputenc}
  \usepackage{textcomp} % provide euro and other symbols
\else % if luatex or xetex
  \usepackage{unicode-math} % this also loads fontspec
  \defaultfontfeatures{Scale=MatchLowercase}
  \defaultfontfeatures[\rmfamily]{Ligatures=TeX,Scale=1}
\fi
\usepackage{lmodern}
\ifPDFTeX\else
  % xetex/luatex font selection
\fi
% Use upquote if available, for straight quotes in verbatim environments
\IfFileExists{upquote.sty}{\usepackage{upquote}}{}
\IfFileExists{microtype.sty}{% use microtype if available
  \usepackage[]{microtype}
  \UseMicrotypeSet[protrusion]{basicmath} % disable protrusion for tt fonts
}{}
\makeatletter
\@ifundefined{KOMAClassName}{% if non-KOMA class
  \IfFileExists{parskip.sty}{%
    \usepackage{parskip}
  }{% else
    \setlength{\parindent}{0pt}
    \setlength{\parskip}{6pt plus 2pt minus 1pt}}
}{% if KOMA class
  \KOMAoptions{parskip=half}}
\makeatother
\usepackage{xcolor}
\usepackage[margin=1in]{geometry}
\usepackage{color}
\usepackage{fancyvrb}
\newcommand{\VerbBar}{|}
\newcommand{\VERB}{\Verb[commandchars=\\\{\}]}
\DefineVerbatimEnvironment{Highlighting}{Verbatim}{commandchars=\\\{\}}
% Add ',fontsize=\small' for more characters per line
\usepackage{framed}
\definecolor{shadecolor}{RGB}{248,248,248}
\newenvironment{Shaded}{\begin{snugshade}}{\end{snugshade}}
\newcommand{\AlertTok}[1]{\textcolor[rgb]{0.94,0.16,0.16}{#1}}
\newcommand{\AnnotationTok}[1]{\textcolor[rgb]{0.56,0.35,0.01}{\textbf{\textit{#1}}}}
\newcommand{\AttributeTok}[1]{\textcolor[rgb]{0.13,0.29,0.53}{#1}}
\newcommand{\BaseNTok}[1]{\textcolor[rgb]{0.00,0.00,0.81}{#1}}
\newcommand{\BuiltInTok}[1]{#1}
\newcommand{\CharTok}[1]{\textcolor[rgb]{0.31,0.60,0.02}{#1}}
\newcommand{\CommentTok}[1]{\textcolor[rgb]{0.56,0.35,0.01}{\textit{#1}}}
\newcommand{\CommentVarTok}[1]{\textcolor[rgb]{0.56,0.35,0.01}{\textbf{\textit{#1}}}}
\newcommand{\ConstantTok}[1]{\textcolor[rgb]{0.56,0.35,0.01}{#1}}
\newcommand{\ControlFlowTok}[1]{\textcolor[rgb]{0.13,0.29,0.53}{\textbf{#1}}}
\newcommand{\DataTypeTok}[1]{\textcolor[rgb]{0.13,0.29,0.53}{#1}}
\newcommand{\DecValTok}[1]{\textcolor[rgb]{0.00,0.00,0.81}{#1}}
\newcommand{\DocumentationTok}[1]{\textcolor[rgb]{0.56,0.35,0.01}{\textbf{\textit{#1}}}}
\newcommand{\ErrorTok}[1]{\textcolor[rgb]{0.64,0.00,0.00}{\textbf{#1}}}
\newcommand{\ExtensionTok}[1]{#1}
\newcommand{\FloatTok}[1]{\textcolor[rgb]{0.00,0.00,0.81}{#1}}
\newcommand{\FunctionTok}[1]{\textcolor[rgb]{0.13,0.29,0.53}{\textbf{#1}}}
\newcommand{\ImportTok}[1]{#1}
\newcommand{\InformationTok}[1]{\textcolor[rgb]{0.56,0.35,0.01}{\textbf{\textit{#1}}}}
\newcommand{\KeywordTok}[1]{\textcolor[rgb]{0.13,0.29,0.53}{\textbf{#1}}}
\newcommand{\NormalTok}[1]{#1}
\newcommand{\OperatorTok}[1]{\textcolor[rgb]{0.81,0.36,0.00}{\textbf{#1}}}
\newcommand{\OtherTok}[1]{\textcolor[rgb]{0.56,0.35,0.01}{#1}}
\newcommand{\PreprocessorTok}[1]{\textcolor[rgb]{0.56,0.35,0.01}{\textit{#1}}}
\newcommand{\RegionMarkerTok}[1]{#1}
\newcommand{\SpecialCharTok}[1]{\textcolor[rgb]{0.81,0.36,0.00}{\textbf{#1}}}
\newcommand{\SpecialStringTok}[1]{\textcolor[rgb]{0.31,0.60,0.02}{#1}}
\newcommand{\StringTok}[1]{\textcolor[rgb]{0.31,0.60,0.02}{#1}}
\newcommand{\VariableTok}[1]{\textcolor[rgb]{0.00,0.00,0.00}{#1}}
\newcommand{\VerbatimStringTok}[1]{\textcolor[rgb]{0.31,0.60,0.02}{#1}}
\newcommand{\WarningTok}[1]{\textcolor[rgb]{0.56,0.35,0.01}{\textbf{\textit{#1}}}}
\usepackage{graphicx}
\makeatletter
\def\maxwidth{\ifdim\Gin@nat@width>\linewidth\linewidth\else\Gin@nat@width\fi}
\def\maxheight{\ifdim\Gin@nat@height>\textheight\textheight\else\Gin@nat@height\fi}
\makeatother
% Scale images if necessary, so that they will not overflow the page
% margins by default, and it is still possible to overwrite the defaults
% using explicit options in \includegraphics[width, height, ...]{}
\setkeys{Gin}{width=\maxwidth,height=\maxheight,keepaspectratio}
% Set default figure placement to htbp
\makeatletter
\def\fps@figure{htbp}
\makeatother
\setlength{\emergencystretch}{3em} % prevent overfull lines
\providecommand{\tightlist}{%
  \setlength{\itemsep}{0pt}\setlength{\parskip}{0pt}}
\setcounter{secnumdepth}{-\maxdimen} % remove section numbering
\ifLuaTeX
  \usepackage{selnolig}  % disable illegal ligatures
\fi
\IfFileExists{bookmark.sty}{\usepackage{bookmark}}{\usepackage{hyperref}}
\IfFileExists{xurl.sty}{\usepackage{xurl}}{} % add URL line breaks if available
\urlstyle{same}
\hypersetup{
  pdftitle={20074027 STAT2003 TASK 2},
  pdfauthor={Bruce Odek Omondi},
  hidelinks,
  pdfcreator={LaTeX via pandoc}}

\title{20074027 STAT2003 TASK 2}
\author{Bruce Odek Omondi}
\date{2025-05-12}

\begin{document}
\maketitle

\hypertarget{question-a}{%
\section{QUESTION A}\label{question-a}}

\begin{enumerate}
\def\labelenumi{\roman{enumi}.}
\item
  Using an effective experimental design, this experiment aimed to study
  and evaluate the antiviral effect of different combinations of drugs
  (the 5 drugs in the study) against a virus (HSV-1), AND to detect any
  interactions among the different drug components.
\item
  The experimental factors are each of the 5 drugs(A,B,C,D,E): A -
  Alpha, B - beta, C - gamma (Interferon drugs) D - Ribavirin, E -
  Acyclovir (Chemical drugs)
\item
  There are 3 levels for each experimental factor, i.e., each drug. Each
  one has a different purpose: High (+1) - maximum dosage from pilot
  study Middle (0) - 32 times diluted from high dosage Low (-1) - no
  drug
\end{enumerate}

Having these 3 levels allows the researchers to capture and estimate not
only the linear and quadratic effects, but also the interaction effect
among the drug combinations better than a 2-level design, and this is
crucial for the purpose of understanding how the drugs work together.

\begin{enumerate}
\def\labelenumi{\roman{enumi}.}
\setcounter{enumi}{3}
\tightlist
\item
  The treatments are the combinations of the different drugs at
  different dosage levels as shown in each run. The researchers did this
  using a composite design where:
\end{enumerate}

\begin{itemize}
\tightlist
\item
  16 treatments have 3 levels encoded as -1 and 1 to match their dosage
  level, allowing for estimation of linear and interaction effects.
\item
  18 treatments have 3 levels encoded as 0,-1, 1 to match their dosage
  level. This allowed estimation of linear, quadratic, and interaction
  effects.
\item
  So, there were 34 total treatments (runs) applied to the infected
  cells.
\end{itemize}

\begin{enumerate}
\def\labelenumi{\alph{enumi}.}
\setcounter{enumi}{21}
\tightlist
\item
  The response is the readout, i.e., the percentage of cells infected
  with the virus. A lower response value indicates better outcome in the
  drug trial, i.e., the drugs are more effective at reducing the viral
  infection.
\end{enumerate}

\begin{enumerate}
\def\labelenumi{\roman{enumi}.}
\setcounter{enumi}{5}
\tightlist
\item
  The basic principles of experiment design used in this study are:
\end{enumerate}

\begin{enumerate}
\def\labelenumi{\arabic{enumi}.}
\tightlist
\item
  Blocking - althought implicit, the use of a single batch for all
  experiments to reduce variation from different batches indicates the
  implementation of blocking
\item
  Replication - each treatment was carried out in 2 separate runs to
  account for variability
\item
  Randomization - the 2 independent researchers used random orders AND
  the 34 runs were randomly assigned to the wells in the plates to avoid
  bias
\item
  Control - the study mentioned a run where there was viral infection
  but no drug application, which can be used for comparison, indicating
  the implementation of a control in the experiment
\end{enumerate}

\hypertarget{question-b}{%
\section{QUESTION B}\label{question-b}}

\hypertarget{part-1-analysis-to-replicate-estimates-ab-and-c-in-table-iii-using-all-34-runs}{%
\subsection{1. Part 1: Analysis to replicate estimates a,b, and c in
Table III using all 34
runs}\label{part-1-analysis-to-replicate-estimates-ab-and-c-in-table-iii-using-all-34-runs}}

Opening the dataset

\begin{Shaded}
\begin{Highlighting}[]
\CommentTok{\# Load required packages}
\FunctionTok{library}\NormalTok{(tidyverse)}
\end{Highlighting}
\end{Shaded}

\begin{verbatim}
## -- Attaching core tidyverse packages ------------------------ tidyverse 2.0.0 --
## v dplyr     1.1.2     v readr     2.1.4
## v forcats   1.0.0     v stringr   1.5.1
## v ggplot2   3.4.4     v tibble    3.2.1
## v lubridate 1.9.2     v tidyr     1.3.0
## v purrr     1.0.1     
## -- Conflicts ------------------------------------------ tidyverse_conflicts() --
## x dplyr::filter() masks stats::filter()
## x dplyr::lag()    masks stats::lag()
## i Use the conflicted package (<http://conflicted.r-lib.org/>) to force all conflicts to become errors
\end{verbatim}

\begin{Shaded}
\begin{Highlighting}[]
\CommentTok{\# Read data}
\NormalTok{drugdata }\OtherTok{\textless{}{-}} \FunctionTok{read.csv}\NormalTok{(}\StringTok{"antiviraldrugs.csv"}\NormalTok{)}
\end{Highlighting}
\end{Shaded}

I. Estimates in Table III (a) with y=readout and all runs included

\begin{Shaded}
\begin{Highlighting}[]
\CommentTok{\# Add average readout (no removal of any outliers, no transformation)}
\NormalTok{drugs\_a }\OtherTok{\textless{}{-}}\NormalTok{ drugdata }\SpecialCharTok{\%\textgreater{}\%}
  \FunctionTok{mutate}\NormalTok{(}\AttributeTok{avg\_readout =}\NormalTok{ (Replicate1 }\SpecialCharTok{+}\NormalTok{ Replicate2) }\SpecialCharTok{/} \DecValTok{2}\NormalTok{,}
         \AttributeTok{replicate =} \FunctionTok{rep}\NormalTok{(}\FunctionTok{c}\NormalTok{(}\SpecialCharTok{{-}}\DecValTok{1}\NormalTok{, }\DecValTok{1}\NormalTok{), }\AttributeTok{each =} \DecValTok{17}\NormalTok{))}

\CommentTok{\# Fit full second{-}order model using raw response}
\NormalTok{model\_a }\OtherTok{\textless{}{-}} \FunctionTok{lm}\NormalTok{(avg\_readout }\SpecialCharTok{\textasciitilde{}}\NormalTok{ A }\SpecialCharTok{+}\NormalTok{ B }\SpecialCharTok{+}\NormalTok{ C }\SpecialCharTok{+}\NormalTok{ D }\SpecialCharTok{+}\NormalTok{ E }\SpecialCharTok{+}
                \FunctionTok{I}\NormalTok{(A}\SpecialCharTok{\^{}}\DecValTok{2}\NormalTok{) }\SpecialCharTok{+} \FunctionTok{I}\NormalTok{(B}\SpecialCharTok{\^{}}\DecValTok{2}\NormalTok{) }\SpecialCharTok{+} \FunctionTok{I}\NormalTok{(C}\SpecialCharTok{\^{}}\DecValTok{2}\NormalTok{) }\SpecialCharTok{+} \FunctionTok{I}\NormalTok{(D}\SpecialCharTok{\^{}}\DecValTok{2}\NormalTok{) }\SpecialCharTok{+} \FunctionTok{I}\NormalTok{(E}\SpecialCharTok{\^{}}\DecValTok{2}\NormalTok{) }\SpecialCharTok{+}
\NormalTok{                A}\SpecialCharTok{:}\NormalTok{B }\SpecialCharTok{+}\NormalTok{ A}\SpecialCharTok{:}\NormalTok{C }\SpecialCharTok{+}\NormalTok{ A}\SpecialCharTok{:}\NormalTok{D }\SpecialCharTok{+}\NormalTok{ A}\SpecialCharTok{:}\NormalTok{E }\SpecialCharTok{+}
\NormalTok{                B}\SpecialCharTok{:}\NormalTok{C }\SpecialCharTok{+}\NormalTok{ B}\SpecialCharTok{:}\NormalTok{D }\SpecialCharTok{+}\NormalTok{ B}\SpecialCharTok{:}\NormalTok{E }\SpecialCharTok{+}
\NormalTok{                C}\SpecialCharTok{:}\NormalTok{D }\SpecialCharTok{+}\NormalTok{ C}\SpecialCharTok{:}\NormalTok{E }\SpecialCharTok{+}\NormalTok{ D}\SpecialCharTok{:}\NormalTok{E }\SpecialCharTok{+}
\NormalTok{                replicate,}
              \AttributeTok{data =}\NormalTok{ drugs\_a)}

\CommentTok{\# Summary of model (matches Table III(a))}
\FunctionTok{summary}\NormalTok{(model\_a)}
\end{Highlighting}
\end{Shaded}

\begin{verbatim}
## 
## Call:
## lm(formula = avg_readout ~ A + B + C + D + E + I(A^2) + I(B^2) + 
##     I(C^2) + I(D^2) + I(E^2) + A:B + A:C + A:D + A:E + B:C + 
##     B:D + B:E + C:D + C:E + D:E + replicate, data = drugs_a)
## 
## Residuals:
##     Min      1Q  Median      3Q     Max 
## -7.1366 -2.6036 -0.8743  2.4686  8.4523 
## 
## Coefficients:
##              Estimate Std. Error t value Pr(>|t|)    
## (Intercept)  17.87589    6.76783   2.641  0.02152 *  
## A            -1.59155    1.29155  -1.232  0.24144    
## B            -2.35092    1.29155  -1.820  0.09374 .  
## C            -2.00099    1.29155  -1.549  0.14727    
## D           -19.57614    1.29155 -15.157 3.45e-09 ***
## E           -13.93498    1.29155 -10.789 1.57e-07 ***
## I(A^2)        3.52854    3.38347   1.043  0.31756    
## I(B^2)        0.04055    3.38347   0.012  0.99063    
## I(C^2)       -1.82319    3.38347  -0.539  0.59984    
## I(D^2)       -7.33321    3.38347  -2.167  0.05103 .  
## I(E^2)       15.27194    3.38347   4.514  0.00071 ***
## replicate    -0.17232    1.83306  -0.094  0.92665    
## A:B           0.71654    1.40509   0.510  0.61933    
## A:C           2.33747    1.40509   1.664  0.12207    
## A:D           1.46124    1.40502   1.040  0.31883    
## A:E          -1.32398    1.40502  -0.942  0.36461    
## B:C           1.63732    1.40502   1.165  0.26652    
## B:D          -0.27370    1.40509  -0.195  0.84881    
## B:E           1.30919    1.40502   0.932  0.36981    
## C:D          -0.65668    1.40502  -0.467  0.64860    
## C:E           0.11626    1.40509   0.083  0.93542    
## D:E           9.56342    1.40509   6.806 1.89e-05 ***
## ---
## Signif. codes:  0 '***' 0.001 '**' 0.01 '*' 0.05 '.' 0.1 ' ' 1
## 
## Residual standard error: 6.504 on 12 degrees of freedom
## Multiple R-squared:  0.9762, Adjusted R-squared:  0.9345 
## F-statistic: 23.42 on 21 and 12 DF,  p-value: 9.551e-07
\end{verbatim}

\begin{enumerate}
\def\labelenumi{\Roman{enumi}.}
\setcounter{enumi}{1}
\tightlist
\item
  Estimates in Table III (b) with y= sqrt(readout) and all runs included
\end{enumerate}

\begin{Shaded}
\begin{Highlighting}[]
\CommentTok{\# Create data with sqrt(readout) but still including outlier}
\NormalTok{drugs\_b }\OtherTok{\textless{}{-}}\NormalTok{ drugdata }\SpecialCharTok{\%\textgreater{}\%}
  \FunctionTok{mutate}\NormalTok{(}\AttributeTok{avg\_readout =}\NormalTok{ (Replicate1 }\SpecialCharTok{+}\NormalTok{ Replicate2) }\SpecialCharTok{/} \DecValTok{2}\NormalTok{,}
         \AttributeTok{sqrt\_readout =} \FunctionTok{sqrt}\NormalTok{(avg\_readout),}
         \AttributeTok{replicate =} \FunctionTok{rep}\NormalTok{(}\FunctionTok{c}\NormalTok{(}\SpecialCharTok{{-}}\DecValTok{1}\NormalTok{, }\DecValTok{1}\NormalTok{), }\AttributeTok{each =} \DecValTok{17}\NormalTok{))}

\CommentTok{\# Fit model using sqrt(readout)}
\NormalTok{model\_b }\OtherTok{\textless{}{-}} \FunctionTok{lm}\NormalTok{(sqrt\_readout }\SpecialCharTok{\textasciitilde{}}\NormalTok{ A }\SpecialCharTok{+}\NormalTok{ B }\SpecialCharTok{+}\NormalTok{ C }\SpecialCharTok{+}\NormalTok{ D }\SpecialCharTok{+}\NormalTok{ E }\SpecialCharTok{+}
                \FunctionTok{I}\NormalTok{(A}\SpecialCharTok{\^{}}\DecValTok{2}\NormalTok{) }\SpecialCharTok{+} \FunctionTok{I}\NormalTok{(B}\SpecialCharTok{\^{}}\DecValTok{2}\NormalTok{) }\SpecialCharTok{+} \FunctionTok{I}\NormalTok{(C}\SpecialCharTok{\^{}}\DecValTok{2}\NormalTok{) }\SpecialCharTok{+} \FunctionTok{I}\NormalTok{(D}\SpecialCharTok{\^{}}\DecValTok{2}\NormalTok{) }\SpecialCharTok{+} \FunctionTok{I}\NormalTok{(E}\SpecialCharTok{\^{}}\DecValTok{2}\NormalTok{) }\SpecialCharTok{+}
\NormalTok{                A}\SpecialCharTok{:}\NormalTok{B }\SpecialCharTok{+}\NormalTok{ A}\SpecialCharTok{:}\NormalTok{C }\SpecialCharTok{+}\NormalTok{ A}\SpecialCharTok{:}\NormalTok{D }\SpecialCharTok{+}\NormalTok{ A}\SpecialCharTok{:}\NormalTok{E }\SpecialCharTok{+}
\NormalTok{                B}\SpecialCharTok{:}\NormalTok{C }\SpecialCharTok{+}\NormalTok{ B}\SpecialCharTok{:}\NormalTok{D }\SpecialCharTok{+}\NormalTok{ B}\SpecialCharTok{:}\NormalTok{E }\SpecialCharTok{+}
\NormalTok{                C}\SpecialCharTok{:}\NormalTok{D }\SpecialCharTok{+}\NormalTok{ C}\SpecialCharTok{:}\NormalTok{E }\SpecialCharTok{+}\NormalTok{ D}\SpecialCharTok{:}\NormalTok{E }\SpecialCharTok{+}
\NormalTok{                replicate,}
              \AttributeTok{data =}\NormalTok{ drugs\_b)}

\CommentTok{\# Summary of model (matches Table III(b))}
\FunctionTok{summary}\NormalTok{(model\_b)}
\end{Highlighting}
\end{Shaded}

\begin{verbatim}
## 
## Call:
## lm(formula = sqrt_readout ~ A + B + C + D + E + I(A^2) + I(B^2) + 
##     I(C^2) + I(D^2) + I(E^2) + A:B + A:C + A:D + A:E + B:C + 
##     B:D + B:E + C:D + C:E + D:E + replicate, data = drugs_b)
## 
## Residuals:
##      Min       1Q   Median       3Q      Max 
## -0.70275 -0.18779 -0.04771  0.17332  0.77122 
## 
## Coefficients:
##              Estimate Std. Error t value Pr(>|t|)    
## (Intercept)  4.045619   0.568363   7.118 1.22e-05 ***
## A           -0.133744   0.108464  -1.233 0.241159    
## B           -0.230536   0.108464  -2.125 0.054998 .  
## C           -0.200151   0.108464  -1.845 0.089798 .  
## D           -2.059020   0.108464 -18.983 2.56e-10 ***
## E           -1.217984   0.108464 -11.229 1.01e-07 ***
## I(A^2)       0.244341   0.284144   0.860 0.406686    
## I(B^2)       0.074547   0.284144   0.262 0.797497    
## I(C^2)      -0.001232   0.284144  -0.004 0.996612    
## I(D^2)      -1.172732   0.284144  -4.127 0.001402 ** 
## I(E^2)       1.393339   0.284144   4.904 0.000364 ***
## replicate   -0.012681   0.153941  -0.082 0.935707    
## A:B          0.124954   0.118000   1.059 0.310481    
## A:C          0.259862   0.118000   2.202 0.047945 *  
## A:D          0.072360   0.117994   0.613 0.551156    
## A:E         -0.130754   0.117994  -1.108 0.289515    
## B:C          0.142028   0.117994   1.204 0.251915    
## B:D         -0.093080   0.118000  -0.789 0.445524    
## B:E          0.129679   0.117994   1.099 0.293313    
## C:D         -0.105543   0.117994  -0.894 0.388652    
## C:E          0.053797   0.118000   0.456 0.656596    
## D:E          0.535907   0.118000   4.542 0.000676 ***
## ---
## Signif. codes:  0 '***' 0.001 '**' 0.01 '*' 0.05 '.' 0.1 ' ' 1
## 
## Residual standard error: 0.5462 on 12 degrees of freedom
## Multiple R-squared:  0.9815, Adjusted R-squared:  0.9492 
## F-statistic: 30.37 on 21 and 12 DF,  p-value: 2.17e-07
\end{verbatim}

\begin{enumerate}
\def\labelenumi{\Roman{enumi}.}
\setcounter{enumi}{2}
\tightlist
\item
  Estimates in Table III (c) with y= sqrt(readout) and run 14 of
  Replicate 1 removed
\end{enumerate}

\begin{Shaded}
\begin{Highlighting}[]
\CommentTok{\# Clean the dataset: remove replicate 1 of run 14, compute averages, apply transformation, encode replicate variable}
\NormalTok{drugs\_c }\OtherTok{\textless{}{-}}\NormalTok{ drugdata }\SpecialCharTok{\%\textgreater{}\%}
  \FunctionTok{mutate}\NormalTok{(}\AttributeTok{Replicate1 =} \FunctionTok{ifelse}\NormalTok{(Run }\SpecialCharTok{==} \DecValTok{14}\NormalTok{, }\ConstantTok{NA}\NormalTok{, Replicate1)) }\SpecialCharTok{\%\textgreater{}\%}
  \FunctionTok{mutate}\NormalTok{(}\AttributeTok{avg\_readout =} \FunctionTok{rowMeans}\NormalTok{(}\FunctionTok{select}\NormalTok{(., Replicate1, Replicate2), }\AttributeTok{na.rm =} \ConstantTok{TRUE}\NormalTok{),}
         \AttributeTok{sqrt\_readout =} \FunctionTok{sqrt}\NormalTok{(avg\_readout),}
         \AttributeTok{replicate =} \FunctionTok{rep}\NormalTok{(}\FunctionTok{c}\NormalTok{(}\SpecialCharTok{{-}}\DecValTok{1}\NormalTok{, }\DecValTok{1}\NormalTok{), }\AttributeTok{each =} \DecValTok{17}\NormalTok{))}

\CommentTok{\# Fit the full second{-}order model}
\NormalTok{model\_c }\OtherTok{\textless{}{-}} \FunctionTok{lm}\NormalTok{(sqrt\_readout }\SpecialCharTok{\textasciitilde{}}\NormalTok{ A }\SpecialCharTok{+}\NormalTok{ B }\SpecialCharTok{+}\NormalTok{ C }\SpecialCharTok{+}\NormalTok{ D }\SpecialCharTok{+}\NormalTok{ E }\SpecialCharTok{+}
                \FunctionTok{I}\NormalTok{(A}\SpecialCharTok{\^{}}\DecValTok{2}\NormalTok{) }\SpecialCharTok{+} \FunctionTok{I}\NormalTok{(B}\SpecialCharTok{\^{}}\DecValTok{2}\NormalTok{) }\SpecialCharTok{+} \FunctionTok{I}\NormalTok{(C}\SpecialCharTok{\^{}}\DecValTok{2}\NormalTok{) }\SpecialCharTok{+} \FunctionTok{I}\NormalTok{(D}\SpecialCharTok{\^{}}\DecValTok{2}\NormalTok{) }\SpecialCharTok{+} \FunctionTok{I}\NormalTok{(E}\SpecialCharTok{\^{}}\DecValTok{2}\NormalTok{) }\SpecialCharTok{+}
\NormalTok{                A}\SpecialCharTok{:}\NormalTok{B }\SpecialCharTok{+}\NormalTok{ A}\SpecialCharTok{:}\NormalTok{C }\SpecialCharTok{+}\NormalTok{ A}\SpecialCharTok{:}\NormalTok{D }\SpecialCharTok{+}\NormalTok{ A}\SpecialCharTok{:}\NormalTok{E }\SpecialCharTok{+}
\NormalTok{                B}\SpecialCharTok{:}\NormalTok{C }\SpecialCharTok{+}\NormalTok{ B}\SpecialCharTok{:}\NormalTok{D }\SpecialCharTok{+}\NormalTok{ B}\SpecialCharTok{:}\NormalTok{E }\SpecialCharTok{+}
\NormalTok{                C}\SpecialCharTok{:}\NormalTok{D }\SpecialCharTok{+}\NormalTok{ C}\SpecialCharTok{:}\NormalTok{E }\SpecialCharTok{+}\NormalTok{ D}\SpecialCharTok{:}\NormalTok{E }\SpecialCharTok{+}
\NormalTok{                replicate,}
              \AttributeTok{data =}\NormalTok{ drugs\_c)}

\CommentTok{\# View model summary (this reproduces Table III(c))}
\FunctionTok{summary}\NormalTok{(model\_c)}
\end{Highlighting}
\end{Shaded}

\begin{verbatim}
## 
## Call:
## lm(formula = sqrt_readout ~ A + B + C + D + E + I(A^2) + I(B^2) + 
##     I(C^2) + I(D^2) + I(E^2) + A:B + A:C + A:D + A:E + B:C + 
##     B:D + B:E + C:D + C:E + D:E + replicate, data = drugs_c)
## 
## Residuals:
##     Min      1Q  Median      3Q     Max 
## -0.5020 -0.1979 -0.0292  0.1546  0.5969 
## 
## Coefficients:
##             Estimate Std. Error t value Pr(>|t|)    
## (Intercept)  4.02584    0.47738   8.433 2.18e-06 ***
## A           -0.10962    0.09110  -1.203 0.252052    
## B           -0.26608    0.09110  -2.921 0.012823 *  
## C           -0.22067    0.09110  -2.422 0.032183 *  
## D           -2.03490    0.09110 -22.337 3.83e-11 ***
## E           -1.23851    0.09110 -13.595 1.19e-08 ***
## I(A^2)       0.25507    0.23866   1.069 0.306197    
## I(B^2)       0.06670    0.23866   0.279 0.784649    
## I(C^2)      -0.01213    0.23866  -0.051 0.960285    
## I(D^2)      -1.16200    0.23866  -4.869 0.000386 ***
## I(E^2)       1.38244    0.23866   5.793 8.57e-05 ***
## replicate    0.01778    0.12930   0.137 0.892924    
## A:B          0.16523    0.09911   1.667 0.121348    
## A:C          0.28653    0.09911   2.891 0.013552 *  
## A:D          0.04496    0.09911   0.454 0.658203    
## A:E         -0.09556    0.09911  -0.964 0.353968    
## B:C          0.10195    0.09911   1.029 0.323922    
## B:D         -0.05280    0.09911  -0.533 0.603937    
## B:E          0.08960    0.09911   0.904 0.383750    
## C:D         -0.07035    0.09911  -0.710 0.491375    
## C:E          0.02021    0.09911   0.204 0.841847    
## D:E          0.56257    0.09911   5.676 0.000103 ***
## ---
## Signif. codes:  0 '***' 0.001 '**' 0.01 '*' 0.05 '.' 0.1 ' ' 1
## 
## Residual standard error: 0.4588 on 12 degrees of freedom
## Multiple R-squared:  0.9869, Adjusted R-squared:  0.9639 
## F-statistic: 42.96 on 21 and 12 DF,  p-value: 2.923e-08
\end{verbatim}

\begin{Shaded}
\begin{Highlighting}[]
\CommentTok{\# Diagnostic plots to check model assumptions}
\FunctionTok{par}\NormalTok{(}\AttributeTok{mfrow =} \FunctionTok{c}\NormalTok{(}\DecValTok{2}\NormalTok{, }\DecValTok{2}\NormalTok{))}
\FunctionTok{plot}\NormalTok{(model\_c)}
\end{Highlighting}
\end{Shaded}

\includegraphics{Omondi-2074027-Task-2_files/figure-latex/unnamed-chunk-4-1.pdf}

Performing variable selection via stepwise regression to ensure there
are no further significant effects before we re-plot the diagnostic
plots to check for model assumptions.

\begin{Shaded}
\begin{Highlighting}[]
\FunctionTok{library}\NormalTok{(MASS)}
\end{Highlighting}
\end{Shaded}

\begin{verbatim}
## 
## Attaching package: 'MASS'
\end{verbatim}

\begin{verbatim}
## The following object is masked from 'package:dplyr':
## 
##     select
\end{verbatim}

\begin{Shaded}
\begin{Highlighting}[]
\CommentTok{\# Stepwise regression to select significant effects}
\NormalTok{step\_model }\OtherTok{\textless{}{-}} \FunctionTok{stepAIC}\NormalTok{(model\_c, }\AttributeTok{direction =} \StringTok{"both"}\NormalTok{, }\AttributeTok{trace =} \ConstantTok{FALSE}\NormalTok{)}

\CommentTok{\# Step 6: View final model summary}
\FunctionTok{summary}\NormalTok{(step\_model)}
\end{Highlighting}
\end{Shaded}

\begin{verbatim}
## 
## Call:
## lm(formula = sqrt_readout ~ A + B + C + D + E + I(A^2) + I(D^2) + 
##     I(E^2) + A:B + A:C + A:E + B:C + D:E, data = drugs_c)
## 
## Residuals:
##     Min      1Q  Median      3Q     Max 
## -0.4590 -0.2182 -0.1234  0.2554  0.6367 
## 
## Coefficients:
##             Estimate Std. Error t value Pr(>|t|)    
## (Intercept)  4.09019    0.22218  18.409 5.22e-14 ***
## A           -0.10242    0.07341  -1.395  0.17827    
## B           -0.27317    0.07371  -3.706  0.00140 ** 
## C           -0.21201    0.07360  -2.880  0.00925 ** 
## D           -2.04714    0.07421 -27.585  < 2e-16 ***
## E           -1.21788    0.07382 -16.499 4.10e-13 ***
## I(A^2)       0.24187    0.17887   1.352  0.19140    
## I(D^2)      -1.14291    0.17969  -6.361 3.31e-06 ***
## I(E^2)       1.35297    0.17909   7.555 2.79e-07 ***
## A:B          0.15727    0.08096   1.943  0.06628 .  
## A:C          0.29607    0.08096   3.657  0.00157 ** 
## A:E         -0.10175    0.08060  -1.262  0.22133    
## B:C          0.10690    0.08166   1.309  0.20537    
## D:E          0.56397    0.08189   6.887 1.09e-06 ***
## ---
## Signif. codes:  0 '***' 0.001 '**' 0.01 '*' 0.05 '.' 0.1 ' ' 1
## 
## Residual standard error: 0.3821 on 20 degrees of freedom
## Multiple R-squared:  0.9848, Adjusted R-squared:  0.975 
## F-statistic: 99.81 on 13 and 20 DF,  p-value: 2.989e-15
\end{verbatim}

\begin{Shaded}
\begin{Highlighting}[]
\CommentTok{\# Diagnostic plots}
\FunctionTok{par}\NormalTok{(}\AttributeTok{mfrow =} \FunctionTok{c}\NormalTok{(}\DecValTok{2}\NormalTok{, }\DecValTok{2}\NormalTok{))}
\FunctionTok{plot}\NormalTok{(step\_model)}
\end{Highlighting}
\end{Shaded}

\includegraphics{Omondi-2074027-Task-2_files/figure-latex/unnamed-chunk-5-1.pdf}

\hypertarget{part-2-analysis-for-16-run-design-only}{%
\subsection{2. Part 2: Analysis for 16-run design
only}\label{part-2-analysis-for-16-run-design-only}}

\begin{Shaded}
\begin{Highlighting}[]
\CommentTok{\# Subset only the first 16 runs (2{-}level factorial design)}
\NormalTok{drugdata\_16 }\OtherTok{\textless{}{-}}\NormalTok{ drugdata[}\DecValTok{1}\SpecialCharTok{:}\DecValTok{16}\NormalTok{, ]}

\DocumentationTok{\#\# a) Model using raw readout (no transformation, no outlier removed)}
\NormalTok{drugs16\_a }\OtherTok{\textless{}{-}}\NormalTok{ drugdata\_16 }\SpecialCharTok{\%\textgreater{}\%}
  \FunctionTok{mutate}\NormalTok{(}\AttributeTok{avg\_readout =}\NormalTok{ (Replicate1 }\SpecialCharTok{+}\NormalTok{ Replicate2) }\SpecialCharTok{/} \DecValTok{2}\NormalTok{,}
         \AttributeTok{replicate =} \FunctionTok{rep}\NormalTok{(}\FunctionTok{c}\NormalTok{(}\SpecialCharTok{{-}}\DecValTok{1}\NormalTok{, }\DecValTok{1}\NormalTok{), }\AttributeTok{each =} \DecValTok{8}\NormalTok{))}

\NormalTok{model16\_a }\OtherTok{\textless{}{-}} \FunctionTok{lm}\NormalTok{(avg\_readout }\SpecialCharTok{\textasciitilde{}}\NormalTok{ A }\SpecialCharTok{+}\NormalTok{ B }\SpecialCharTok{+}\NormalTok{ C }\SpecialCharTok{+}\NormalTok{ D }\SpecialCharTok{+}\NormalTok{ E }\SpecialCharTok{+}
\NormalTok{                  A}\SpecialCharTok{:}\NormalTok{B }\SpecialCharTok{+}\NormalTok{ A}\SpecialCharTok{:}\NormalTok{C }\SpecialCharTok{+}\NormalTok{ A}\SpecialCharTok{:}\NormalTok{D }\SpecialCharTok{+}\NormalTok{ A}\SpecialCharTok{:}\NormalTok{E }\SpecialCharTok{+}
\NormalTok{                  B}\SpecialCharTok{:}\NormalTok{C }\SpecialCharTok{+}\NormalTok{ B}\SpecialCharTok{:}\NormalTok{D }\SpecialCharTok{+}\NormalTok{ B}\SpecialCharTok{:}\NormalTok{E }\SpecialCharTok{+}
\NormalTok{                  C}\SpecialCharTok{:}\NormalTok{D }\SpecialCharTok{+}\NormalTok{ C}\SpecialCharTok{:}\NormalTok{E }\SpecialCharTok{+}\NormalTok{ D}\SpecialCharTok{:}\NormalTok{E }\SpecialCharTok{+}
\NormalTok{                  replicate,}
                \AttributeTok{data =}\NormalTok{ drugs16\_a)}

\FunctionTok{summary}\NormalTok{(model16\_a)}
\end{Highlighting}
\end{Shaded}

\begin{verbatim}
## 
## Call:
## lm(formula = avg_readout ~ A + B + C + D + E + A:B + A:C + A:D + 
##     A:E + B:C + B:D + B:E + C:D + C:E + D:E + replicate, data = drugs16_a)
## 
## Residuals:
## ALL 16 residuals are 0: no residual degrees of freedom!
## 
## Coefficients: (1 not defined because of singularities)
##             Estimate Std. Error t value Pr(>|t|)
## (Intercept)  27.8469        NaN     NaN      NaN
## A            -2.9906        NaN     NaN      NaN
## B            -2.5469        NaN     NaN      NaN
## C             0.5375        NaN     NaN      NaN
## D           -19.6250        NaN     NaN      NaN
## E           -10.9687        NaN     NaN      NaN
## replicate    -2.7062        NaN     NaN      NaN
## A:B               NA         NA      NA       NA
## A:C           0.8094        NaN     NaN      NaN
## A:D           1.9969        NaN     NaN      NaN
## A:E          -1.3219        NaN     NaN      NaN
## B:C           2.4781        NaN     NaN      NaN
## B:D           0.2781        NaN     NaN      NaN
## B:E           1.1719        NaN     NaN      NaN
## C:D          -0.8906        NaN     NaN      NaN
## C:E           0.1656        NaN     NaN      NaN
## D:E           9.0156        NaN     NaN      NaN
## 
## Residual standard error: NaN on 0 degrees of freedom
## Multiple R-squared:      1,  Adjusted R-squared:    NaN 
## F-statistic:   NaN on 15 and 0 DF,  p-value: NA
\end{verbatim}

\begin{Shaded}
\begin{Highlighting}[]
\DocumentationTok{\#\# b) Model using sqrt(readout), no outlier removed}
\NormalTok{drugs16\_b }\OtherTok{\textless{}{-}}\NormalTok{ drugdata\_16 }\SpecialCharTok{\%\textgreater{}\%}
  \FunctionTok{mutate}\NormalTok{(}\AttributeTok{avg\_readout =}\NormalTok{ (Replicate1 }\SpecialCharTok{+}\NormalTok{ Replicate2) }\SpecialCharTok{/} \DecValTok{2}\NormalTok{,}
         \AttributeTok{sqrt\_readout =} \FunctionTok{sqrt}\NormalTok{(avg\_readout),}
         \AttributeTok{replicate =} \FunctionTok{rep}\NormalTok{(}\FunctionTok{c}\NormalTok{(}\SpecialCharTok{{-}}\DecValTok{1}\NormalTok{, }\DecValTok{1}\NormalTok{), }\AttributeTok{each =} \DecValTok{8}\NormalTok{))}

\NormalTok{model16\_b }\OtherTok{\textless{}{-}} \FunctionTok{lm}\NormalTok{(sqrt\_readout }\SpecialCharTok{\textasciitilde{}}\NormalTok{ A }\SpecialCharTok{+}\NormalTok{ B }\SpecialCharTok{+}\NormalTok{ C }\SpecialCharTok{+}\NormalTok{ D }\SpecialCharTok{+}\NormalTok{ E }\SpecialCharTok{+}
\NormalTok{                  A}\SpecialCharTok{:}\NormalTok{B }\SpecialCharTok{+}\NormalTok{ A}\SpecialCharTok{:}\NormalTok{C }\SpecialCharTok{+}\NormalTok{ A}\SpecialCharTok{:}\NormalTok{D }\SpecialCharTok{+}\NormalTok{ A}\SpecialCharTok{:}\NormalTok{E }\SpecialCharTok{+}
\NormalTok{                  B}\SpecialCharTok{:}\NormalTok{C }\SpecialCharTok{+}\NormalTok{ B}\SpecialCharTok{:}\NormalTok{D }\SpecialCharTok{+}\NormalTok{ B}\SpecialCharTok{:}\NormalTok{E }\SpecialCharTok{+}
\NormalTok{                  C}\SpecialCharTok{:}\NormalTok{D }\SpecialCharTok{+}\NormalTok{ C}\SpecialCharTok{:}\NormalTok{E }\SpecialCharTok{+}\NormalTok{ D}\SpecialCharTok{:}\NormalTok{E }\SpecialCharTok{+}
\NormalTok{                  replicate,}
                \AttributeTok{data =}\NormalTok{ drugs16\_b)}

\FunctionTok{summary}\NormalTok{(model16\_b)}
\end{Highlighting}
\end{Shaded}

\begin{verbatim}
## 
## Call:
## lm(formula = sqrt_readout ~ A + B + C + D + E + A:B + A:C + A:D + 
##     A:E + B:C + B:D + B:E + C:D + C:E + D:E + replicate, data = drugs16_b)
## 
## Residuals:
## ALL 16 residuals are 0: no residual degrees of freedom!
## 
## Coefficients: (1 not defined because of singularities)
##             Estimate Std. Error t value Pr(>|t|)
## (Intercept)  4.62587        NaN     NaN      NaN
## A           -0.28113        NaN     NaN      NaN
## B           -0.28603        NaN     NaN      NaN
## C            0.01286        NaN     NaN      NaN
## D           -2.00331        NaN     NaN      NaN
## E           -0.95730        NaN     NaN      NaN
## replicate   -0.28371        NaN     NaN      NaN
## A:B               NA         NA      NA       NA
## A:C          0.15356        NaN     NaN      NaN
## A:D          0.17699        NaN     NaN      NaN
## A:E         -0.12178        NaN     NaN      NaN
## B:C          0.26711        NaN     NaN      NaN
## B:D         -0.07783        NaN     NaN      NaN
## B:E          0.13509        NaN     NaN      NaN
## C:D         -0.13066        NaN     NaN      NaN
## C:E          0.06971        NaN     NaN      NaN
## D:E          0.50335        NaN     NaN      NaN
## 
## Residual standard error: NaN on 0 degrees of freedom
## Multiple R-squared:      1,  Adjusted R-squared:    NaN 
## F-statistic:   NaN on 15 and 0 DF,  p-value: NA
\end{verbatim}

\begin{Shaded}
\begin{Highlighting}[]
\DocumentationTok{\#\# c) Model using sqrt(readout), with outlier removed (Run 14 replicate 1 removed)}
\NormalTok{drugs16\_c }\OtherTok{\textless{}{-}}\NormalTok{ drugdata\_16 }\SpecialCharTok{\%\textgreater{}\%}
  \FunctionTok{mutate}\NormalTok{(}\AttributeTok{avg\_readout =} \FunctionTok{case\_when}\NormalTok{(}
\NormalTok{    Run }\SpecialCharTok{==} \DecValTok{14} \SpecialCharTok{\textasciitilde{}}\NormalTok{ Replicate2,}
    \ConstantTok{TRUE} \SpecialCharTok{\textasciitilde{}}\NormalTok{ (Replicate1 }\SpecialCharTok{+}\NormalTok{ Replicate2) }\SpecialCharTok{/} \DecValTok{2}
\NormalTok{  ),}
  \AttributeTok{sqrt\_readout =} \FunctionTok{sqrt}\NormalTok{(avg\_readout),}
  \AttributeTok{replicate =} \FunctionTok{rep}\NormalTok{(}\FunctionTok{c}\NormalTok{(}\SpecialCharTok{{-}}\DecValTok{1}\NormalTok{, }\DecValTok{1}\NormalTok{), }\AttributeTok{each =} \DecValTok{8}\NormalTok{))}

\NormalTok{model16\_c }\OtherTok{\textless{}{-}} \FunctionTok{lm}\NormalTok{(sqrt\_readout }\SpecialCharTok{\textasciitilde{}}\NormalTok{ A }\SpecialCharTok{+}\NormalTok{ B }\SpecialCharTok{+}\NormalTok{ C }\SpecialCharTok{+}\NormalTok{ D }\SpecialCharTok{+}\NormalTok{ E }\SpecialCharTok{+}
\NormalTok{                  A}\SpecialCharTok{:}\NormalTok{B }\SpecialCharTok{+}\NormalTok{ A}\SpecialCharTok{:}\NormalTok{C }\SpecialCharTok{+}\NormalTok{ A}\SpecialCharTok{:}\NormalTok{D }\SpecialCharTok{+}\NormalTok{ A}\SpecialCharTok{:}\NormalTok{E }\SpecialCharTok{+}
\NormalTok{                  B}\SpecialCharTok{:}\NormalTok{C }\SpecialCharTok{+}\NormalTok{ B}\SpecialCharTok{:}\NormalTok{D }\SpecialCharTok{+}\NormalTok{ B}\SpecialCharTok{:}\NormalTok{E }\SpecialCharTok{+}
\NormalTok{                  C}\SpecialCharTok{:}\NormalTok{D }\SpecialCharTok{+}\NormalTok{ C}\SpecialCharTok{:}\NormalTok{E }\SpecialCharTok{+}\NormalTok{ D}\SpecialCharTok{:}\NormalTok{E }\SpecialCharTok{+}
\NormalTok{                  replicate,}
                \AttributeTok{data =}\NormalTok{ drugs16\_c)}

\FunctionTok{summary}\NormalTok{(model16\_c)}
\end{Highlighting}
\end{Shaded}

\begin{verbatim}
## 
## Call:
## lm(formula = sqrt_readout ~ A + B + C + D + E + A:B + A:C + A:D + 
##     A:E + B:C + B:D + B:E + C:D + C:E + D:E + replicate, data = drugs16_c)
## 
## Residuals:
## ALL 16 residuals are 0: no residual degrees of freedom!
## 
## Coefficients: (1 not defined because of singularities)
##             Estimate Std. Error t value Pr(>|t|)
## (Intercept)  4.56951        NaN     NaN      NaN
## A           -0.22477        NaN     NaN      NaN
## B           -0.34239        NaN     NaN      NaN
## C            0.01286        NaN     NaN      NaN
## D           -1.89059        NaN     NaN      NaN
## E           -0.95730        NaN     NaN      NaN
## replicate   -0.39643        NaN     NaN      NaN
## A:B               NA         NA      NA       NA
## A:C          0.20992        NaN     NaN      NaN
## A:D          0.12063        NaN     NaN      NaN
## A:E         -0.06542        NaN     NaN      NaN
## B:C          0.21075        NaN     NaN      NaN
## B:D         -0.02147        NaN     NaN      NaN
## B:E          0.07873        NaN     NaN      NaN
## C:D         -0.07430        NaN     NaN      NaN
## C:E          0.01335        NaN     NaN      NaN
## D:E          0.55971        NaN     NaN      NaN
## 
## Residual standard error: NaN on 0 degrees of freedom
## Multiple R-squared:      1,  Adjusted R-squared:    NaN 
## F-statistic:   NaN on 15 and 0 DF,  p-value: NA
\end{verbatim}

Determining the significant effects due to the saturated nature of the
16-run design

\begin{Shaded}
\begin{Highlighting}[]
\CommentTok{\# Load required package for half{-}normal plot}
\CommentTok{\#install.packages("FrF2")  \# Run only once if not already installed}
\FunctionTok{library}\NormalTok{(FrF2)}
\end{Highlighting}
\end{Shaded}

\begin{verbatim}
## Loading required package: DoE.base
\end{verbatim}

\begin{verbatim}
## Loading required package: grid
\end{verbatim}

\begin{verbatim}
## Loading required package: conf.design
\end{verbatim}

\begin{verbatim}
## Registered S3 method overwritten by 'DoE.base':
##   method           from       
##   factorize.factor conf.design
\end{verbatim}

\begin{verbatim}
## 
## Attaching package: 'DoE.base'
\end{verbatim}

\begin{verbatim}
## The following objects are masked from 'package:stats':
## 
##     aov, lm
\end{verbatim}

\begin{verbatim}
## The following object is masked from 'package:graphics':
## 
##     plot.design
\end{verbatim}

\begin{verbatim}
## The following object is masked from 'package:base':
## 
##     lengths
\end{verbatim}

\begin{Shaded}
\begin{Highlighting}[]
\CommentTok{\# Function to create custom half{-}normal plot}
\NormalTok{custom\_half\_normal }\OtherTok{\textless{}{-}} \ControlFlowTok{function}\NormalTok{(model, model\_name) \{}
  \CommentTok{\# Extract coefficients excluding intercept}
\NormalTok{  effects }\OtherTok{\textless{}{-}} \FunctionTok{coef}\NormalTok{(model)[}\SpecialCharTok{{-}}\DecValTok{1}\NormalTok{]}
\NormalTok{  abs\_effects }\OtherTok{\textless{}{-}} \FunctionTok{sort}\NormalTok{(}\FunctionTok{abs}\NormalTok{(effects), }\AttributeTok{decreasing =} \ConstantTok{FALSE}\NormalTok{) }\CommentTok{\#absolute effects for y{-}axis}
  
  \CommentTok{\# Compute half{-}normal quantiles}
\NormalTok{  n }\OtherTok{\textless{}{-}} \FunctionTok{length}\NormalTok{(abs\_effects)}
\NormalTok{  quantiles }\OtherTok{\textless{}{-}} \FunctionTok{qnorm}\NormalTok{((}\DecValTok{1}\SpecialCharTok{:}\NormalTok{n }\SpecialCharTok{{-}} \FloatTok{0.5}\NormalTok{) }\SpecialCharTok{/}\NormalTok{ n)}
  
  \CommentTok{\# Plot}
  \FunctionTok{plot}\NormalTok{(quantiles, abs\_effects,}
       \AttributeTok{main =} \FunctionTok{paste}\NormalTok{(}\StringTok{"Half{-}Normal Plot –"}\NormalTok{, model\_name),}
       \AttributeTok{xlab =} \StringTok{"Normal Quantiles"}\NormalTok{,}
       \AttributeTok{ylab =} \StringTok{"Absolute Effect"}\NormalTok{,}
       \AttributeTok{pch =} \DecValTok{19}\NormalTok{)}
  \FunctionTok{text}\NormalTok{(quantiles, abs\_effects, }\AttributeTok{labels =} \FunctionTok{names}\NormalTok{(abs\_effects), }\AttributeTok{pos =} \DecValTok{3}\NormalTok{, }\AttributeTok{cex =} \FloatTok{0.7}\NormalTok{)}
  \FunctionTok{abline}\NormalTok{(}\FunctionTok{lm}\NormalTok{(abs\_effects }\SpecialCharTok{\textasciitilde{}}\NormalTok{ quantiles), }\AttributeTok{col =} \StringTok{"blue"}\NormalTok{, }\AttributeTok{lty =} \DecValTok{2}\NormalTok{)}
\NormalTok{\}}

\CommentTok{\# Plot all three }
\FunctionTok{custom\_half\_normal}\NormalTok{(model16\_a, }\StringTok{"Model 16a"}\NormalTok{)}
\end{Highlighting}
\end{Shaded}

\includegraphics{Omondi-2074027-Task-2_files/figure-latex/unnamed-chunk-10-1.pdf}

\begin{Shaded}
\begin{Highlighting}[]
\FunctionTok{custom\_half\_normal}\NormalTok{(model16\_b, }\StringTok{"Model 16b"}\NormalTok{)}
\end{Highlighting}
\end{Shaded}

\includegraphics{Omondi-2074027-Task-2_files/figure-latex/unnamed-chunk-11-1.pdf}

\begin{Shaded}
\begin{Highlighting}[]
\FunctionTok{custom\_half\_normal}\NormalTok{(model16\_c, }\StringTok{"Model 16c"}\NormalTok{)}
\end{Highlighting}
\end{Shaded}

\includegraphics{Omondi-2074027-Task-2_files/figure-latex/unnamed-chunk-12-1.pdf}

\end{document}
